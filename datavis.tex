\documentclass[11pt,]{scrartcl}
\usepackage[a4paper, left=2.5cm, right=2.5cm, top=2.5cm, bottom=2.5cm]{geometry}
\newcommand*{\authorfont}{\fontfamily{phv}\selectfont}
\usepackage{lmodern}
\usepackage{abstract}
\renewcommand{\abstractname}{}    % clear the title
\renewcommand{\absnamepos}{empty} % originally center
\newcommand{\blankline}{\quad\pagebreak[2]}
\newenvironment{cslreferences}{}{}
\providecommand{\tightlist}{%
   \setlength{\itemsep}{0pt}\setlength{\parskip}{0pt}} 
\usepackage{longtable,booktabs}

\usepackage{parskip}

%\usepackage{titling}
%\setlength{\droptitle}{-.25cm}

%\setlength{\parindent}{0pt}
%\setlength{\parskip}{6pt plus 2pt minus 1pt}
%\setlength{\emergencystretch}{3em}  % prevent overfull lines 

\usepackage[T1]{fontenc}
\usepackage[utf8]{inputenc}

\usepackage{graphicx}

\usepackage{fancyhdr}
\pagestyle{fancy}
\usepackage{lastpage}
\renewcommand{\headrulewidth}{0.4pt}
\renewcommand{\footrulewidth}{0.0pt} 
\lhead{}
\chead{}
\rhead{\footnotesize \textbf{Datenvisualisierung mit
R} -- Wintersemester 2024/25}
\lfoot{}
\cfoot{\small \thepage/\pageref*{LastPage}}
\rfoot{}

\fancypagestyle{firststyle}
{
\renewcommand{\headrulewidth}{0pt}%
   \fancyhf{}
   \fancyfoot[C]{\small \thepage/\pageref*{LastPage}}
   \fancyhead[R]{\includegraphics[width=5cm]{C:/Users/scarmin/OneDrive - JGU/Lehre/logo_polwi.jpg}}
}
\setlength{\headheight}{46.91196pt}
%\def\labelitemi{--}
%\usepackage{enumitem}
%\setitemize[0]{leftmargin=25pt}
%\setenumerate[0]{leftmargin=25pt}




\makeatletter
\@ifpackageloaded{hyperref}{}{%
\ifxetex
  \usepackage[setpagesize=false, % page size defined by xetex
	unicode=false, % unicode breaks when used with xetex
	xetex]{hyperref}
\else
  \usepackage[unicode=true]{hyperref}
 \fi
}
\@ifpackageloaded{color}{
	\PassOptionsToPackage{usenames,dvipsnames}{color}
}{%
	\usepackage[usenames,dvipsnames]{color}
}
\makeatother
\hypersetup{breaklinks=true,
			bookmarks=true,
	pdfauthor={ ()},
	pdfkeywords = {},  
	pdftitle={\textbf{Datenvisualisierung mit R}},
	colorlinks=true,
	citecolor=blue,
	urlcolor=blue,
	linkcolor=magenta,
	pdfborder={0 0 0}}
\urlstyle{same}  % don't use monospace font for urls


\setcounter{secnumdepth}{0}





\usepackage{setspace}

\title{\textbf{Datenvisualisierung mit R}}
\author{\emph{Armin Schäfer}}
\date{Wintersemester 2024/25}

\begin{document}  
	
		\maketitle
		
	
	\thispagestyle{firststyle}
%	\thispagestyle{empty}
	
	
	\noindent \begin{tabular*}{\textwidth}{ @{\extracolsep{\fill}} lr @{\extracolsep{\fill}}}
		
		
	E-Mail: \texttt{\href{mailto:schaefer@politik.uni-mainz.de}{\nolinkurl{schaefer@politik.uni-mainz.de}}} & Uhrzeit: 8.00-10.00
Uhr\\
	Raum: GFG 01-512 & Büro: GFG 04-436\\
		\hline
	\end{tabular*}
	
	\vspace{2mm}
	
	
	
	\section{Beschreibung}\label{beschreibung}

Inhalt: Der kompetente Umgang mit Daten und deren grafisch Darstellung
werden immer wichtiger. Wissenschaftliche Ergebnisse werden immer
häufiger grafisch dargestellt und nicht nur im Text beschrieben. Auch im
Journalismus gibt es zunehmend spezialisierte Teams, die sich
ausschließlich mit Datenvisulisierung beschäftigen. Im Zentrum dieses
Forschungsseminars steht deshalb die Arbeit mit dem kostenfreien
Statistikprogramm ``R''. Alle Teilnehmer:innen werden lernen, wie sich
Daten mit diesem Programm bearbeiten und auswerten lassen, um Analysen
durchführen und grafische Darstellungen erstellen zu können. Dabei gehen
wir streng \emph{anwendungsorientiert} vor, denn dieses
Forschungsseminar wird keine Einführung in die Statistik ersetzen. Ziel
ist es, verschiedene Varianten der Datenvisualisierung zu erlernen, die
Sie im weiteren Studium nutzen können. Um teilnehmen zu können, sind
keine Vorkenntnisse notwendig.

\section{Aktive Mitarbeit}\label{aktive-mitarbeit}

Um den Lernerfolg sicherzustellen, müssen Sie eine Übungsaufgabe
bearbeiten und den Lösungsweg in Moodle hochladen. Diese Abgabe ist
\emph{verpflichtend}.

\section{Ablauf}\label{ablauf}

\textbf{17. April: Einführung}

\textbf{24. April: Grundlagen}

Vorab lesen: Kapitel 1 aus
\href{https://socviz.co/lookatdata.html\#lookatdata}{Data Visualization.
A practical introduction}.

\begin{enumerate}
\def\labelenumi{\alph{enumi})}
\tightlist
\item
  R, RStudio und Quarto
\item
  Projekte anlegen \& mit Skripten arbeiten
\item
  Bibliotheken installieren und aufrufen
\item
  Erstes Beispiel: Ein simulierter Datensatz
\item
  Datensätze inspizieren
\item
  Die erste Grafik!
\end{enumerate}

\textbf{8. Mai: Balkendiagramme}

Vorab lesen: Kapitel 6 aus
\href{https://clauswilke.com/dataviz/visualizing-amounts.html}{Fundamentals
of Data Visualization}.

\begin{enumerate}
\def\labelenumi{\alph{enumi})}
\tightlist
\item
  Excel-Daten laden: csv und xlsx
\item
  Balkendiagramm nach Parteizugehörigkeit
\item
  Den Parteien Farben zuordnen
\item
  Langes Datenformat
\item
  Balkendiagramme nach Partei und Geschlecht
\end{enumerate}

\textbf{15. Mai: Gestaltungsarten -- die ``Geome''}

Beispiel: Abgeordnete im Bundestag

\begin{enumerate}
\def\labelenumi{\alph{enumi})}
\tightlist
\item
  Punktediagramm: Durchschnittliches Alter
\item
  Alle Datenpunkte zeigen: Jitter
\item
  Verteilungen anzeigen: Histogramm und Density plot
\item
  Alpha effektiv einsetzen
\item
  Das Erscheinungsbild ändern
\end{enumerate}

\textbf{22. Mai: Übungswoche}

Bitte bearbeiten Sie bis zum \textcolor{Orange}{2. Juni (18:00 Uhr)} die
in Moodle beschriebene Übungsaufgabe und laden Sie Ihr Skript als
html-Dokument in den dafür vorgesehenen Ordner hoch. Nach Ablauf der
Frist lade ich eine (einfache) \textcolor{Orange}{Musterlösung} hoch,
die Woche danach lernen wir weitere Schritte, um das Skript weniger
fehleranfällig zu machen.

\textbf{5. Juni: Daten umformen und weitere Darstellungsarten}

\begin{enumerate}
\def\labelenumi{\alph{enumi})}
\tightlist
\item
  Variablen auswählen, filtern und umbenennen
\item
  Variablen verändern: mutate(), case\_when() und if\_else()
\item
  Missings
\item
  Mit Gruppen arbeiten
\end{enumerate}

\textbf{12. Juni: Funktionen I: Unnötige Wiederholungen vermeiden}

Never type the same code twice.

\begin{enumerate}
\def\labelenumi{\alph{enumi})}
\tightlist
\item
  Funktionen erstellen
\item
  Funktionen auf mehrere Spalten anwenden
\item
  Eine ggplot-Funktion
\item
  Ein eigenes ``theme'' erstellen
\end{enumerate}

\textbf{26. Juni: Funktionen II: Grafiken automatisiert erstellen}

\begin{enumerate}
\def\labelenumi{\alph{enumi})}
\tightlist
\item
  Mehr zu \texttt{mutate(across())}
\item
  Eine Funktion für Grafiken
\item
  Grafiken kombinieren
\item
  Ein ``theme'' selbst erstellen
\end{enumerate}

\textbf{3. Juli: Trends im Zeitverlauf}

Beispiel: Autokratisierung

\begin{enumerate}
\def\labelenumi{\alph{enumi})}
\tightlist
\item
  Trends in wenigen Ländern
\item
  Trends in vielen Ländern pro Jahrzehnt
\item
  Dumbbell plots
\item
  Faceting
\item
  Highlighting Beispiel: Wahlbeteiligung und Parteiergebnisse in den
  Wahlkreisen
\end{enumerate}

\textbf{10. Juli: Lineare Regressionsmodelle und Interaktionen grafisch
darstellen}

Beispiel: Wo schneiden Parteien gut oder schlecht ab?

\begin{enumerate}
\def\labelenumi{\alph{enumi})}
\tightlist
\item
  Das unersetzliche Streudiagramm
\item
  Trendlinien hinzufügen
\item
  Koeffizienten als Abbildung
\item
  Interaktionen einfügen
\item
  Interaktionen darstellen
\end{enumerate}

\textbf{17. Juli: Logistische Regression und vorhergesagte
Wahrscheinlichkeiten}

Beispiel: Wer wählt die Grünen?

\begin{enumerate}
\def\labelenumi{\alph{enumi})}
\tightlist
\item
  Logistische Regressionen
\item
  Koeffizientenplots
\item
  Vorhergesagte Wahrscheinlichkeiten
\item
  Marginale Effekte
\end{enumerate}
	
	
	
	
			\end{document}

\makeatletter
\def\@maketitle{%
	\newpage
	%  \null
	%  \vskip 2em%
	%  \begin{center}%
	\let \footnote \thanks
	{\fontsize{12}{10}\selectfont\raggedright  \setlength{\parindent}{0pt} \@title \par}%
}
%\fi
\makeatother
